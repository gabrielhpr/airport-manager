\documentclass{article}
\usepackage[T1]{fontenc}
\usepackage[utf8]{inputenc}
\usepackage[portuguese]{babel}

\title{Relatório EP0}
\author{Gabriel Henrique Pinheiro Rodrigues NUSP: 112.216-47}
\date{February 2020}

\begin{document}
   \maketitle
\section{Implementação}
Na implementação do programa de gerenciamento de um aeroporto, foi utilizado duas classes principais:
\\\\
A \textbf{classe Avião}, que contém:

\begin{itemize}
 \item O nome da companhia aérea;
 \item O número do avião;
 \item Se a intenção é pousar ou decolar;
 \item Se é um voo de emergência ou não;
 \item O nome do aeroporto de origem ou destino;
 \item A quantidade de combustível;
 \item A duração do voo;
 \item O tempo de espera;
\end{itemize}
E a \textbf{classe Aeroporto}, que contém :

\begin{itemize}
 \item O número de elementos da fila;
 \item A disponibilidade de cada uma das 3 pistas;
 \item Ponteiro para a cabeça da fila e para o último elemento;
 \item Métodos para a manipulação da fila;
 \item Métodos para a atualização da disponibilidade das pistas;
\end{itemize}

Por motivos de encapsulamento os elementos da fila são estruturas que apontam para aviões, essas estruturas também possuem o endereço do próximo elemento da fila e para o anterior, dessa forma a fila implementada é uma lista duplamente encadeada;

Os métodos que manipulam a fila são: 

\begin{itemize}
 \item insereNoFim;
 \item insereNoInicio;
 \item remove;
 \item printaElemento;
 \item printaFila;
\end{itemize}
Em tese, uma fila sempre remove do início e insere no fim, porém devido a prioridade de cada voo, foram criados dois métodos de inserção.

\section{Prioridades na fila}
Cada objeto da classe Avião contém um número de 0 a 3 que define sua prioridade na fila, quanto maior o número, maior a prioridade. Os aviões e suas respectivas prioridades são definidas da seguinte forma:

\begin{itemize}
\item 3 - Aviões que estão sem combustível;
\item 2 - Voos de emergência;
\item 1 - Aviões que estão esperando para decolar por mais de 10\% do tempo estimado da duração da viagem;
\item 0 - Demais aviões;
\end{itemize}

A cada unidade de tempo as posições na fila são remanejadas, de forma que os aviões são organizados por ordem descrescente de prioridade.

\section{Testes}
Para realizar os testes escolhemos K=10, T=15, C=30, porém o programa funciona bem para casos maiores.

\end{document}