\documentclass{article}
\usepackage[T1]{fontenc}
\usepackage[utf8]{inputenc}
\usepackage[portuguese]{babel}

\title{Relatório EP0}
\author{Gabriel Henrique Pinheiro Rodrigues NUSP: 112.216-47}
\date{Março 2020}

\begin{document}
   \maketitle

   \section{Como executar o EP na linha de comando: }
Após compilado : ./ep0 T K C V S
\\
onde:
\begin{itemize}
   \item T - int - informa o tempo da simulação;
   \item K - int - número máximo de solicitações a torre de comando por unidade de tempo;
   \item C - int - número máximo de combustível;
   \item V - int - número máximo da duração do voo;
   \item S - int - semente da simulação;
   
\end{itemize}

\section{Plano de Implementação}
Na implementação do programa de gerenciamento de um aeroporto, foram utilizadas duas classes principais: A 
\textbf{classe Avião} e a \textbf{classe Aeroporto}, dentro da classe Aeroporto há 4 filas, cada uma separada pelo nível de
prioridade (*). Também foram criados \textbf{2 structs}, um chamado \textbf{elementoDaFila}, que contém
um ponteiro para o objeto da classe Avião e é utilizado como célula da fila, e outro struct chamado
\textbf{cabecaTorreDeControle} que contém ponteiros para os elementos iniciais e finais de cada uma das
4 filas;
\\\\
(*) As \textbf{prioridades} foram definidas da seguinte forma:
\begin{itemize}
   \item 3 - Prioridade máxima - Corresponde aos aviões sem combustível;
   \item 2 - Voos de emergência - Ex. presidencial, acidentes;
   \item 1 - Aviões que já estão esperando por mais de 10\% do tempo total de voo para decolar;
   \item 0 - Demais aviões;   
\end{itemize}

A \textbf{classe Avião} contém:

\begin{itemize}
 \item O nome da companhia aérea;
 \item O número do avião;
 \item Se a intenção é pousar ou decolar;
 \item Se é um voo de emergência ou não;
 \item O nome do aeroporto de origem ou destino;
 \item A quantidade de combustível;
 \item A duração do voo;
 \item O tempo de espera;
 \item A prioridade do voo;
\end{itemize}

A \textbf{classe Aeroporto} contém:

\begin{itemize}
 \item O número de elementos da fila;
 \item A disponibilidade de cada uma das 3 pistas;
 \item Ponteiro para a cabecaTorreDeControle - Struct que contém ponteiros que apontam para os 
 elementos iniciais e finais de cada uma das 4 filas;
 \item Métodos para a manipulação da fila;
 \item Métodos para a atualização da disponibilidade das pistas;
 \item Métodos de exibição dos elementos da fila;
\end{itemize}

Por motivos de encapsulamento os elementos da fila são estruturas que apontam para aviões, essas estruturas também possuem o endereço do próximo elemento da fila e para o anterior, dessa forma a fila implementada é uma lista duplamente encadeada;

\section{Funcionamento do Aeroporto}
A cada unidade de tempo os métodos da classe Aeroporto que são executados são:

\begin{itemize}
   \item \textbf{contatoComATorre(Aviao* aviao)} - Verifica se é possível inserir o avião em alguma fila ou se
   é melhor que esse seja direcionado a outro aeroporto (*1);
   \item \textbf{liberaVoos()} - Enquanto houver pistas disponíveis tenta liberar os aviões para pouso e decolagem
   respeitando os níveis de prioridade;
   \item \textbf{atualizaSituacaoDosAvioesNaFila()} - Decrementa a quantidade de combustível dos aviões na fila, aumenta
   o tempo de espera e altera a prioridade caso necessário. (Caso algum avião de prioridade 0
   fica sem combustível, sua prioridade vai para 3, se o avião já estiver esperando por mais de 10\% do
   seu tempo de voo, sua prioridade vai para 1);
   \item \textbf{atualizaDispDePista()} - Decrementa o tempo de espera para a liberação das pistas e troca a sua
   disponibilidade;
   \item \textbf{coletaEstatisticasEPrinta()} - Armazena dados interessantes e exibe na hora da execução - Ex. número de aviões que cairam,
   número de aviões enviados a outros aeroportos, ...
\end{itemize}
(*1) Critérios utilizados:\\\\
 - Se já há 9 emergências na fila e o avião atual é de pouso de emergência esse será enviado a outro aeroporto;
\\\\
 - Se ainda não há 9 emergências na fila e o avião atual é de pouso de emergência, mas sua quantidade de combustível
 é menor que o tempo que os outros aviões de emergência vão levar para liberar a pista, então ele será
 enviado a outro aeroporto; 
\\\\
 - Se o avião tem prioridade 0, mas sua quantidade de combustível é menor que o tempo de liberação
da pista, então ele também é enviado a outro aeroporto;
\\\\
 - Caso contrário, ele é inserido em sua respectiva fila(De acordo com a sua prioridade);


\section{Testes}
Para gerar os objetos da classe Avião criamos uma função que monta de maneira semi-aleatória o objeto, a
\textbf{proporção de voos de emergência é 10\%}, foram escolhidas \textbf{7 companhias aéreas} e \textbf{33 aeroportos}, a probabilidade
de um avião querer pousar ou decolar é a mesma.

\subsection{Os testes realizados foram}
0-) INPUT: T=500, K=3, C=20, V=20, S=17\\
   
OUTPUT: 
\begin{itemize}
   \item Os aviões que estão esperando para pousar são: Nenhum;
   \item Os aviões que estão esperando para decolar são: Nenhum;
   \item A quantidade média de combustível dos aviões que pousaram: 10.7351;
   \item A quantidade de aviões pousando/decolando em condições de emergência é: 29;
   \item A quantidade total de aviões que já pousaram é: 185;
   \item A quantidade total de aviões que já decolaram é: 201;
   \item A quantidade de aviões que cairam é: 0;
   \item A porcentagem de aviões enviados para outros aeroportos é: 3\%;
\end{itemize}
\\
1-) INPUT: T=500, K=5, C=20, V=20, S=17\\
   
OUTPUT: 
\begin{itemize}
   \item Os aviões que estão esperando para pousar são: Nenhum;
   \item Os aviões que estão esperando para decolar são: Nenhum;
   \item A quantidade média de combustível dos aviões que pousaram: 9.94595;
   \item A quantidade de aviões pousando/decolando em condições de emergência é: 39;
   \item A quantidade total de aviões que já pousaram é: 296;
   \item A quantidade total de aviões que já decolaram é: 270;
   \item A quantidade de aviões que cairam é: 0;
   \item A porcentagem de aviões enviados para outros aeroportos é: 12\%;
\end{itemize}

2-) INPUT: T=500, K=7, C=20, V=20, S=17\\

OUTPUT: 
\begin{itemize}
   \item Os aviões que estão esperando para pousar são: LA465; AZ207; AZ761;
   \item Os aviões que estão esperando para decolar são: AZ207; AZ761;
   \item A quantidade média de combustível dos aviões que pousaram: 4.19298;
   \item A quantidade de aviões pousando/decolando em condições de emergência é: 36;
   \item A quantidade total de aviões que já pousaram é: 228;
   \item A quantidade total de aviões que já decolaram é: 343;
   \item A quantidade de aviões que cairam é: 0;
   \item A porcentagem de aviões enviados para outros aeroportos é: 44\%;
\end{itemize}

\end{document}